\chapter{Az adatok elérése}

Mivel az adatok közvetlenül nem lekérdezhetőek a napelemes honlapról, ezért
megoldást kellett keresnem arra, hogy az adatok hogyan jussanak el a LabVIEW
programhoz. Egy köztes réteget kellett felépítenem a probléma feloldására.
Mivel a korábbiakban már építettem TypeScript segítségével Rest API-t,
ezért a választásom erre a technológiára esett.

Az alkalmazáshoz az express npm csomagot használtam fel, amely egy könnyen
használható és rugalmas keretrendszert kínál Node.js-alapú webalkalmazások
fejlesztéséhez. Az express segítségével gyorsan létrehozhatok útvonalakat,
valamint kezelhetek HTTP kéréseket és válaszokat. Ezen felül a nagy népszerűség
(jelenleg átlagosan heti 25 millió letöltés az \texttt{npmjs.com} oldalról) és a
széles körben elérhető dokumentáció (\texttt{expressjs.com})
segít abban, hogy könnyedén megtaláljam a szükséges információkat és segítséget,
ha bármilyen problémába ütköznék. Ráadásul a typescripthez szükséges típus
definíciók is elérhetőek hozzá.

\section{Általános adatok}

Az alkalmazás skálázhatósága érdekében először létrehoztam a \texttt{`/about`}
útvonalat amely visszaadja, hogy milyen napokon érhetőek el a termelési adatok,
valamint azt is megmondja, hogy hány inverter, és inverterenként hány panel
van telepítve.

Mintaadat:

\begin{minted}{json}
// about.json
{
  "dates": [ "2023-05-24", "2023-05-23", "...", "2023-05-01" ],
  "count": {
    "inverter": 8,
    "panel": 4
  }
}
\end{minted}

Az ehhez tartozó \texttt{`about.d.ts`} deklarációs fájl:

\begin{minted}{typescript}
// about.d.ts
type DateString = `${number}-${number}-${number}`;
type About = {
  dates: Array<DateString>;
  count: {
    inverter: number;
    panel: number;
  };
};
\end{minted}

\section{Panel adatok}

Ezek után az egyes panelokhoz tartozó adatok lekérdezéséhez tartozó
\texttt{`/panel/:date`} útvonalat hoztam létre.

Mintaadat:

\begin{minted}{json}
[
  3011, // panel 01
  3047, // panel 02
  // ..... panel ..
  2595  // panel 32
]
\end{minted}

Az ehhez tartozó \texttt{`panel.d.ts`} deklarációs fájl:

\begin{minted}{typescript}
type Panel = Array<number>;
\end{minted}

\section{Részletes adatok}

Legvégül pedig a rendszer negyedórás bontásban elérhető teljesítmény és
megtermelt energia adataihoz készítettem el a \texttt{`/detailed/:date`}
útvonalat.

Mintaadat:

\begin{minted}{json}
{
  "power": [
    0,
    // ...
    0, 2, 80,
    // ...
    43, 2, 0,
    // ...
  ],
  "sum": [
    0,
    // ...
    0, 0.5, 20.5,
    // ...
    86135.75, 86136.25, 86136.25,
    // ...
  ],
  "hour": [
    "00:00",
    "00:30",
    // ...
    "23:30",
    "23:45"
  ]
}
\end{minted}

Az ehhez tartozó \texttt{`{}detailed.d.ts`} deklarációs fájl:

\begin{minted}{typescript}
type Detailed = {
  power: Array<number>;
  sum: Array<number>;
  hour: Array<string>;
};
\end{minted}

\section{Az útvonalak összefoglalása}

A három elérhető útvonal tehát:

\begin{minted}{typescript}
// Saját middleware-ek importálása
import parseDateMW from '../middleware/parseDate';
import getSinglePanelMW from '../middleware/getSinglePanel';
import getSingleDetailedMW from '../middleware/getSingleDetailed';
import getAbout from '../middleware/getAbout';
import sendDataMW from '../middleware/sendData';

// Szükséges típus importálása
import type { Express } from 'express';

// Az alkalmazás útvonalainak összefoglalása
export default function route(app: Express) {
  app.get(
    '/detailed/:date',     // útvonal
    parseDateMW(),         // :date feldolgozása
    getSingleDetailedMW(), // részletes adatok lekérdezése
    sendDataMW()           // adat küldése
  );
  app.get(
    '/panel/:date',        // útvonal
    parseDateMW(),         // :date feldolgozása
    getSingleDetailedMW(), // panel adatok lekérdezése
    sendDataMW()           // adat küldése
  );
  app.get(
    '/about',              // útvonal
    getAboutMW(),          // általános adatok lekérdezése
    sendDataMW()           // adat küldése
  );
}
\end{minted}

Ezek után még megfelelő parsereket kellett választani, és már készen is
volt az alkalmazás.
\begin{minted}{typescript}
// .env fájl figyelembe vétele
import 'dotenv/config';
// npm csomagok
import express from 'express';
import cors from 'cors';
// Saját modulok
import parse from './parser';
import route from './route';

const app = express();
const PORT = process.env.PORT || 3000;

app.use(cors({ origin: 'http://localhost:3000', credentials: true }));
app.use(express.urlencoded({ extended: true }));
app.use(express.json());
app.use(express.static('static'));

route(app);

app.listen(PORT, () => {
  console.log(`App is listening on port ${PORT}`);
});
\end{minted}

Készen is van a Rest API. Térjünk át a LabVIEW programra.
