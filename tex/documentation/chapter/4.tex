\chapter{Összefoglalás}

Az projektem célja tehát az volt, hogy egy általam fejlesztett API segítségével
termelési adatokat szolgáltassak, majd ezeket az adatokat a LabVIEW fejlesztői
környezetben feldolgozzam.

Az API fejlesztést TypeScript nyelven, az express npm csomag segítségével
oldottam meg, amely egy olyan keretrendszer, amely lehetővé teszi a gyors és
hatékony webes szolgáltatások kialakítását.

Az adatokat ezután a LabVIEW fejlesztői környezetben dolgoztam fel, amely egy
olyan grafikus programozási nyelv, amely lehetővé teszi például mérőrendszerek
fejlesztését. A programozási környezetben olyan elemeket használtam, mint a
diagramok és különböző matematikai és adat manipulációs műveleti blokkok, hogy
feldolgozzam és elemezzem a napelem termelési adatokat.

A LabVIEW segítségével különböző elemzéseket végeztem a napelem termelési
adatokon. Például megjelenítettem a termelés időbeli változását, létrehoztam
grafikonokat és összehasonlítottam az egyes panelokhoz és inverterekhez tartozó
adatokat.

A projekt során fontos szempont volt a megbízhatóság és a hatékonyság. Az
express API biztosította az adatok gyors és megbízható elérését, míg a LabVIEW
lehetővé tette az adatok könnyű feldolgozását és elemzését. Az így létrehozott
rendszer segítségével részletes információkat és jelentéseket lehet készíteni a
napelemek termelésével kapcsolatban.
